\documentclass[twoside]{article}
\usepackage{graphicx}
\usepackage[english]{babel} 
\usepackage{hyperref}
\usepackage{fancyhdr}
\usepackage{geometry}
\usepackage{biblatex}
\usepackage{titlesec}
\usepackage{setspace}

% -------> Arial Format
\usepackage{helvet}
\renewcommand{\familydefault}{\sfdefault}
\titleformat*{\section}{\fontsize{16}{19.2}\bfseries\sffamily}
\titleformat*{\subsection}{\fontsize{14}{16.8}\bfseries\sffamily}
\titleformat*{\subsubsection}{\fontsize{12}{14.4}\bfseries\sffamily} 

% -----> Espaciado entre lineas
\renewcommand{\baselinestretch}{1.5} 


%-----> Justifications of the Document
\usepackage{ragged2e} 
\justifying

\makeatletter
\def\HUGE{\@setfontsize\HUGE{30}{60}}
\setlength{\parskip}{0.1in}

% --------> Information about the Number page and Characteristhics
\usepackage{fancyhdr}
\pagestyle{fancy}
\fancyhf{}

\newcommand{\institute}{Adrián José Villalobos Peraza, Héctor Alonso Caravaca Vargas, Isaac Ramírez Rojas}
\newcommand{\authors}{Current and future Extended Reality applications}
\fancyhead[RE]{\ifnum\value{page}>1 \institute \fi}
\fancyhead[LO]{\ifnum\value{page}>1 \authors \fi}

% Cabecera y pie de página
\usepackage{fancyhdr}
\pagestyle{fancy}
\fancyhf{}
% Definir los encabezados y pie de página
\fancyhead[LO]{\ifnum\value{page}>1 \authors \fi} % Autores a la izquierda en páginas impares
\fancyhead[RE]{\ifnum\value{page}>1 \institute \fi} % Instituto a la derecha en páginas pares
\fancyfoot[RO,LE]{\centering \thepage} % Número de página a la derecha en páginas pares e izquierda en páginas impares
\fancyfootoffset[R]{0.5cm} % Ajustar margen derecho del pie de página
% References
\begin{document}
\begin{titlepage}
\begin{center}
\includegraphics[width=0.15\textwidth]{./logo}\\[1cm]
\rule[1.2cm]{\textwidth}{3pt}\\[0.5cm]
\textsc{\LARGE \textbf{Tecnologico de Costa Rica}}\\[0.5cm]
\textsc{\Large Escuela de Ingenieria en Computacion}\\[0.5cm]
\textsc{\Large Current and future Extended Reality applications}\\[0.5cm]
% 3 participants
\begin{minipage}{0.4\textwidth}
\begin{flushleft} \large
\emph{Participants:A,B,C}
\end{flushleft}
% Professor
\end{minipage}
\begin{minipage}{0.4\textwidth}
\begin{flushright} \large
\emph{Professor: Alfaro Velazco Jorge}
\end{flushright}
\end{minipage}
\vfill


\tableofcontents
\clearpage

\tableofcontents

\section{Abstract}
Here you can put your abstract information 

\clearpage

\section{Introduction}
Here you can put your abstract information

\clearpage

\section{Justification}
Here you can put your justification information

\clearpage

\subsection{Objectives}
    \subsubsection{General Objective}
    Here the General Objetive
    \subsubsection{Specific Objective}
    \begin{itemize}
        \item{First Item}
            Here the 1'st item 
        \item{Second Item}
            Here the 2'nd item
    \end{itemize}
\clearpage
\subsection{Art State}
\subsection{Extended Reality}
\subsubsection{Definition}
Extended reality (XR) stands for a combination of real and virtual enviroment technologies, that covers Aumented Reality (AR) and Mixed Reality (MR). This technology is used to create immersive experiences for the user, by combining real and virtual worlds. The term XR initialized when Charles Wyckoff filled a patent in 1960 for his development. Then nowadays the term has moved to the mainstream of videogames and medicine.(Hayes, 2023)

\begin{itemize}
    \item{Augmented Reality (AR)}

        AR is an enhanced version of the real physical world that is achieved through the use of digital visual elements, sound, or other sensory stimuli and delivered via technology. It is a growing trend among companies involved in mobile computing and business applications in particular. 


    \item{Mixed Reality (MR)}

        MR is the merge of real and virtual worlds to recreate and produce new environments and visualizations where physical and digital objects co-exist one with the other.
        This new reality is based on advancements in computer vision, graphical processing, display technologies, input systems, and cloud computing.

    \item{Virtual Reality (VR)}

        the use of computer modeling and simulation that enables a person to interact with an artificial three-dimensional (3D) visual or other sensory environment.

\end{itemize}
\subsection{Applications}
\subsubsection{Education}
Augmented reality has been helpfull in the education field, but when it really starts to help students and kids is when the Covid 19 begin. When teachers try to teach students with normal methods they get borred and lose easily their concentration, and that's when the AR comes in. When the students learn by using interactions and uses technologies they start to get insterest in the topics the teacher teach.

Integrating augmented reality into the learning process will be helpfull and will bring to the students an interactive dimension that strongly resonates with students, specially in situationes where learning occurs remotely or through a hybrid approach. The fusion of tangible real-world and digitals overlays nurtures an immersive learning enviroment that captivates students attention and incentivates their interests in subjects that are being presented.

\subsubsection{Gaming and entertainment}
Nowadays, augmented reality has gained recognition due to video games and industries that produce technology linked to the VR concept. For example, Meta with the Oculus Quest, PlayStation with its own VR headset, and now Apple joins the battle with the Vision Pro. All these devices have something in common: virtual reality.

Have you ever thought about what you can do in a video game with augmented reality? For example, Racing Simulation, also known as sim racing, is a very popular term when pilots from various categories use a VR headset and a sim racing combo to play and improve their times in games. By training in a virtual environment, they can test things that in real life can be very expensive, such as crashing a car, changing weather conditions, and testing different cars and their behaviors. This helps them improve their times, and their teams, who cover the costs of cars, fuel, tracks, etc., will not face any risks.

Another important use of Augmented Reality is entertainment. For example, using an Oculus Quest, you can easily watch movies, series, and YouTube videos by searching for what you want. It’s very immersive, as all you see is a big screen with the media you’re playing. By combining this with mixed reality, you can adjust the size of the screen and interact with other people in your environment.

Exercise using VR is also a significant topic. In the world, there are many different body types, for example, muscular and slim. Using a VR headset, you can play games that involve moving around, which causes your body to stay active and burn fat. This is important because some people don’t enjoy traditional exercise, making it a great way to stay active and burn fat.

It’s crucial to know that you should use this kind of technology in moderation, as it can be very addictive. It’s recommended not to use it excessively during the day, as it can lead to eye damage and addiction.

\subsubsection{Healthcare}
The integration of digital technologies has leading a new paradigm shift in teaching methods, for example in augmented reality and mixed reality has a potential to transform the medical education, why? because with this kind of technology people can learn the procedures, surgical planning and guidance.

The review of 26 studies that used augmented reality and mixed reality in medical education, found that VR and XR can improve in the training of professionals in health.

One of the key strengths highlighted in the review is the versatility of AR and MR applications across various subjects and learner types. The studies covered diverse healthcare disciplines, such as anatomy and anesthesia, catering to learners of different levels and specialties. These technologies offer an interactive and immersive learning experience, enabling participants to engage with virtual and real-world elements simultaneously. Moreover, the studies showcased a rich diversity in research focus, ranging from introducing novel applications to evaluating their impact on training and learning outcomes.

There are some weaknesses in the VR and XR, for example, a lot of studies showed early versions of things without really checking how helpful they were for learning, as well they said that using AR and MR was really good for improvement and learning, but others didn't find clear proof of this. Because of these mixed results, it's important to have better ways to do the research, like using the same methods for everyone, checking lots of different aspects, and including more people in the studies.



\subsubsection{Architecture and Design}

% Type here

\subsection{Second section}
% Type here

\subsection{Third section}
% Type here
% Add new sections

\subsection{Health and well-being of people}
% Type here health and well-being ER improvements
\subsubsection{Other uses}
Extended reality could also improve the lives of disabled people, for example, despite how odd it might seem, mixed reality can help people affected by blindness or visual impairments. In this case, the approach involves creating a program on a device compatible with spatial sensing such as the HoloLens 2. This program would be capable of recognizing the surroundings and providing auditory alerts about the environment. This is especially useful for people who depends of a white cane, as the cane helps perceive the ground but not the area above their pelvis.

By implementing what was described before, blind individuals could gain the ability to recognize the entire surroundings, giving them more security and protection. Another method to achieve this is through vibrating warnings, for example from a smartwatch that would send more intense warnings depending on how far objects are.


\subsection{Simulations and professional training}
%Type here all releated with simulations or professional training (for example, architecture)
\subsubsection{Medical training}
People has different ways to learn, but one of the best ways to learn is through practice. In other contexts, such engineering or various sciences areas, people can practice by using different tools and laboratories equipped with everything necessary to recreate the same activities they would perform as professionals. However, the landscape changes in medicine. For medical students, the solution nowadays, is to work with anatomical models that simulate the human body. These models, although being beneficial, may overlook some aspects that would happen in real-life scenarios.
\subsubsection{Hospitality and hotel personnel training}
Extended reality also has a significant impact on hotel and tourism industry. Well-trained personnel are indispensable for this section. For example, employees need to interact with customers and provide them the best services. Additionally, because hotel industry is constantly evolving in social, technological and economic terms, ongoing training is of great importance.

To improve hotel personnel training, XR (Extended Reality) could be employed, here are some reasons why:
\begin{itemize}
    \item Through the utilization of this technology, employees can be immersed in various scenarios and exposed to different types of customers they may attend, giving them more experience and knowledge.
    \item All employees are different, some of them could be fast learners, while others need more time to learn. Utilizing XR technologies can facilitate interactive and personalized learning. This way everyone can learn at their own pace.  
    \item Also, hotel personnel can be evaluated by their performance in virtual scenarios provided by XR, allowing hotel superiors to give employees different recommendations and identify areas they should improve 
\end{itemize}

\newpage
\section{References}
\end{document}

